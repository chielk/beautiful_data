\documentclass{article}[10pt]

%\usepackage{fullpage}

\usepackage[english]{babel}
\usepackage{microtype}
\usepackage{graphicx}
\usepackage{caption}
\usepackage{subcaption} \usepackage{hyperref}


\title{Scientific Visualization and Virtual Reality – Exercise 4}
\author{Maarten Inja (5872464) \& Chiel Kooijman (5743028)}

\begin{document}
\maketitle

In this report we describe how we have visualized a frog. We have been given two
processed datasets: one with the exterior (skin) of a frog, and one with the
intestines. The latter set was separated; each organ had an index.

We used VTK to create a visualization. The datatype was given in the
assignment, and so it was not difficult to read the data. We used the index of an
organ throughout our program to specific the color in the representation,
whether an organ should be rendered, and the opacity of organs.

A standard \emph{vtkRenderWindowInteractor} was implemented to enable the default
vtk interaction behaviour of the visualisation. Furthermore, we created a class
which extends the \emph{vtkInteractorStyleTrackballCamera} in order to catch
mouse events. This enabled us to check if a user presses on organ name in the
legend and subsequently disable or renable rendering of the selected organ.

In the process of selecting the colors for the organs we balanced between our
goal of choosing colors that are realistic (i.e. the expected color of the organ
in real frogs) and our goal of creating a rendering that is easy :w

expected color
The colors of the organs were chosen as close to the expected color of organs in
real frogs with



\end{document}
